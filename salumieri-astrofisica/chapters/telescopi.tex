\chapter{Telescopi}
    \epigraph{\itshape C'è una forchetta conficcata nel terreno.}{F. Pinguino}
    \noindent\yinitpar{C}\textsc{ome sappiamo}, la radiazione elettromagnetica è una delle principali fonti di informazioni sugli eventi astronomici e dallo studio della radiazione che giunge sulla Terra dallo spazio è possibile a volte risalire ad alcune proprietà degli eventi astronomici che l'hanno generata.

    Ad esempio, la luce che nel suo tragitto viene riflessa può risultare polarizzata, quella che attraversa gas e polveri può partecipare a fenomeni di scattering e cambiare la propria lunghezza d'onda, così come anche le particelle cariche possono produrre ulteriore radiazione tramite effetto Cherenkov o radiazione di sincrotrone.

    Studiando la radiazione che incide sui nostri strumenti tentiamo quindi di ricostruire il processo fisico che l'ha generata per dedurre le condizioni al contorno che hanno permesso a quel processo di verificarsi.

    Naturalmente questa tecnica ha delle difficoltà legate al fatto che diversi fenomeni possono generare radiazione elettromagnetica simile, e alla limitatezza degli strumenti utilizzati.
    \section{Acquisizione dei dati}
        Per studiare un oggetto nel cielo, la prassi è quella di puntare gli strumenti nella sua direzione e ad un certo istante di tempo misurare l'intensità specifica\footnote{Densità di energia al variare di tutto, vedi Sez xx?}. Per un attimo, immaginiamo di trascurare tutti i fenomeni che alterano la radiazione nel percorso dalla sorgente al rivelatore e concentriamoci sul solo processo di misura.

        Per cominciare, il modo in cui l'osservatore usa lo strumento---che sia consapevole o no delle conseguenze delle proprie scelte---può falsare la misura o causare la perdita di informazioni: se misuro la luminosità di una stella tutti i giorni alla stessa ora e leggo sempre lo stesso valore, potrei essere indotto a pensare che la stella abbia luminosità costante, ma che succede nelle 24 ore di tempo in cui non effettuo misure? La luminosità potrebbe cambiare periodicamente e io potrei aver avuto la ``(s)fortuna'' di aver effettuato le misure in momenti in cui la luminosità assume lo stesso valore, senza pensare che a un orario diverso la luminosità possa essere diversa.

        Assumendo che l'osservatore prenda le misure in modo impeccabile, dovrà comunque scontrarsi con i limiti tecnici dell'apparecchio che ha davanti: il potere risolutivo dell'apparato utilizzato potrebbe non essere sufficiente a risolvere due stelle vicine, inoltre non è detto che esso sia sensibile a tutte le lunghezze d'onda allo stesso modo, così come potrebbe non distinguere lunghezze d'onda vicine. Dovrebbe inoltre essere in grado di misurare la polarizzazione della luce incidente e l'esatto numero di fotoni che incidono sul rivelatore \myetc. Purtroppo uno strumento così versatile ed efficiente non esiste.
    \section{Storia dei telescopi}
    \section{Proprietà geometriche del telescopio}
        \subsection{La distribuzione di Poisson}
    \section{Telescopi rifrattori e riflettori}
    \section{Metodi costruttivi e montature}
    \section{Ottica adattiva}
        \subsection{Seeing}