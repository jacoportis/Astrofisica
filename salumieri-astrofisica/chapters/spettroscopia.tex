\chapter{Spettroscopia}
    La fotometria ci permette di misurare la radiazione elettromagnetica in bande larghe circa \SI{1500}{\angstrom}. Questo ci permette di risalire alle proprietà generali della sorgente che la emette, come la temperatura e la distanza. Tuttavia, come abbiamo visto, tra le sorgenti e i nostri strumenti si trovano oggetti materiali che la luce deve attraversare per raggiungerci.
    
    Proviamo per un attimo a fare un ragionamento al contrario: immaginiamo di avere una sorgente di cui conosciamo tutte le caratteristiche principali---ad esempio proprio un corpo nero---e interponiamo del materiale tra noi e la sorgente. Possiamo ad esempio confrontare la luce che arriva ai nostri strumenti con la curva teorica prodotta dal corpo nero al fine capire come la radiazione interagisce con la materia e dedurre da ciò le sue proprietà.
\section{Assorbimento ed emissione}
    La materia che la radiazione elettromagnetica attraversa dopo aver lasciato la stella è di vario tipo: polveri e grani, gas ionizzati, plasmi, e le stesse atmosfere della stella e del nostro pianeta, ammesso che i nostri strumenti non si trovino nello spazio.

    I fenomeni di interazione che possono verificarsi sono prevalentemente di tre tipi:
    \begin{enumerate}[label=\ding{70}]
        \item \emph{vero assorbimento}, quando il fotone di energia $h\nu$ viene del tutto ``assorbito'' da un atomo che si eccita e la sua energia si trasforma in energia termodinamica;
        \item \emph{vera emissione}, quando un atomo eccitato---da assorbimento o a causa di urti anelastici con altri atomi---emette radiazione diseccitandosi;
        \item \emph{diffusione Compton}, un fenomeno anelastico in cui un fotone interagisce con un elettrone libero, gli cede energia cinetica e viene diffuso a un angolo $\vartheta$ cambiando lunghezza d'onda.
    \end{enumerate}
    Questi sono fenomeni che possono alterare il 
\section{Spettrografi}
    Per poter misurare la radiazione incidente sugli strumenti con le