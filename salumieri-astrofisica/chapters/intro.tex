\chapter{Introduzione}\label{ch:intro}
    \section{Le dificoltà dell'Astrofisica}
        L'Astrofisica è il ramo della Fisica che studia le proprietà fisiche dei corpi celesti. Essa si deve frequentemente confrontare con distanze di molti oridni  di grandezza speriori a quelle della vita di tutti i giorni e per qesto deve affrontare alcuni problemi che le altre branche della fisica non hanno.

        Mentre un fisico può andare in laboratorio e cambiare le condizioni a contorno del proprio espreimento, l'astrofisico non ha questa possibilità: egli deve dedurre le proprietà dell'oggetto di studio solo tramite osservazioni. Questa differenza richiede che vengano fatte delle ipotesi preliminari senza le quali lo studio dei corpi celesti potrebbe risultare illegittimato.
        
        L'ipotesi su cui si fonda tutta l'Astrofisica è quella che le leggi della fisica siano le stesse ovunque nell'universo---anche ad anni luce di distanza---, cosa che naturalmente non è possibile verificare a meno di, per esempio, di inviare tanti piccoli esploratori in tutti i punti dell'universo per verificare che ciò sia vero.

        Una delle principali difficoltà di questa ipotesi è il fatto che le stesse leggi che applichiamo sulla Terra sono solo modelli che in prima approssimazione descrivono e prevedono sufficientemente bene il mondo che ci circonda e non sono necessariamente corrette. Basti pensare alla Teoria della Gravitazione di Newton che spiega bene l'orbita della Luna intorno alla Terra, ma basta allontanarsi di poco da noi per scoprire che l'orbita di Mercurio ha un comportamento inspiegabile secondo la teoria di Newton ma meglio descritto dalla Relatività Generale.

        L'Astrofisica di conseguenza non può che partire da ipotesi simili---ad esempio assumendo che la Relatività Generale sia vera anche nella galassia di Andromeda---per poi eventualmente confrontarsi con i risultati osservativi e proporre correzioni ai modelli.
        
        In modo del tutto simile, anche le scale temporali sono enormemente più grandi rispetto a quelle della vita dell'uomo. Immaginiamo di guardare diverse foto di una famiglia scattate a distanza di dieci anni l'una dall'altra: in una singola foto saremmo in grado di distinguere le persone anziane da quelle giovani, quelle di sesso maschile da quelle di sesso femminile \myetc, così come confrontando due foto consecutive siamo in grado di riconoscere i cambiamenti nella fisionomia degli individui. Invece immaginiamo di scattare una foto al cielo oggi e un'altra tra dieci anni. Quante differenze saremmo in grado di riconoscere? Quanto cambia una stella nel corso di dieci anni se la sua vita media è dell'ordine di grandezza di miliardi di anni? Si tratta di fenomeni che avvengono su tempi scala troppo lunghi rispetto alla vita umana.

        Il problema dell'Astrofisica è proprio questo: è come tentare di comprendere tutto quello che sappiamo sulla razza umana da una sola foto, di decidere come funzionano le cose semplicemente con uno sguardo attento a un'istantanea, ma non finisce qui! L'informazione infatti si trasferisce con una velocità finita, quindi il segnale proveniente da un oggetto lontano impiegherà più tempo ad arrivare e l'oggetto ci apparirà più giovane. È un po' come se ci venisse chiesto di riconoscere la nonna nella foto di famiglia nonostante appaia come la persona più giovane nella foto.

        Quello che quindi si tenta di fare in Astrofisica è proprio cercare di osservare gli oggetti nel modo più dettagliato possibile per poi cercare di risalire al quadro più generale e estrapolare un modello dell'evoluzione dei corpi celesti.
    \section{Scopi dell'astrofisica}
        %
        %
        %
        %
        %%
        %           SKIP!
        %
        %
        %
        %%
        %
    \section{La misura delle grandezze astronomiche}
        L'Astrofisica trova la sua