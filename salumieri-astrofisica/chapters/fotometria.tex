\chapter{Fotometria}\label{ch:fotometria}
\epigraph{La fotometria}{ciao}
\noindent\yinitpar{\color{purple}L}\textsc{a fotometria} è lo studio della radiazione elettromagnetica come detto prima bla bla bla. \lipsum[1]
\section{Intensità, flusso, magnitudini}.
    Introduciamo le grandezze più importanti della fotometria. Come sappiamo, la radiazione elettromagnetica trasporta un'energia; supponiamo di avere quindi della radiazione che attraversa una superficie $\dd A$ il cui vettore normale forma un angolo $\vartheta$ con la direzione di propagazione. Essa, lasciando la superficie ``alle sue spalle'', si manterrà all'interno di un angolo solido $\dd\Omega$ che stacca dalla normale alla superficie lo stesso angolo $\dd\vartheta$. In generale le radiazione può contenere qualisai lunghezza d'onda: consideriamo inizialmente la radiazione nelle frequenze comprese nell'intervallo  $\bqty{\nu,\nu + \dd\nu}$. L'energia infinitesima che la radiazione trasporta nella regione $\dd\Omega$ sarà quindi $\dd E_\nu \propto \dd t \dd\nu \cos\vartheta\dd A$. Chiamiamo \emph{intensità specifica} la costante di proporzionalità, $I_\nu$, e scriviamo:
    \begin{equation}
        \label{intensità-specifica-freq}
        \dd[4] E_\nu = I_{\nu} \dd{\nu} \dd{t} \cos{\vartheta} \dd{A} \dd{\Omega}
        \myperiod
    \end{equation}
    In modo del tutto analogo possiamo fare lo stesso ragionamento decomponendo lo spettro in lunghezza d'onda anziché in frequenza e avremo:
    \begin{equation}
        \label{intensità-specifica-freq}
        \dd[4] E_\lambda = I_\lambda \,\dd\lambda\,\dd t\cos\vartheta\,\dd A \,\dd\Omega
        \myperiod
    \end{equation}
    Integrando su tutte le frequenze otteniamo l'\emph{intensità totale} denotata dalla lettera $I$ e data da
    \begin{equation}
        \label{intensità-totale}
        \dd[3] E = \int_{0}^{+\infty}I_\nu \,\dd\nu\,\dd t\cos\vartheta\,\dd A \,\dd\Omega = I \,\dd t\cos\vartheta\,\dd A \,\dd\Omega
        \myperiod
    \end{equation}
    Invertendo queste relazioni si trova subito:
    \begin{align}
        I_\nu &= \frac{1}{\cos\vartheta}\frac{\dd[4] E_\nu}{\dd{\nu} \dd{t} \dd{A} \dd{\Omega}} \mycomma \\
        I_\lambda &= \frac{1}{\cos\vartheta}\frac{\dd[4] E_\lambda}{\dd{\lambda} \dd{t} \dd{A} \dd{\Omega}}
        \mycomma \\
        I &= \frac{1}{\cos\vartheta}\frac{\dd[3] E}{\dd t \dd{A} \dd{\Omega}}
        \myperiod
    \end{align}
    Un'altra grandezza utile nell'Astrofisica è il \emph{flusso di energia}, detto atrimenti \emph{flusso} o \emph{luminosità} che coincide con la potenza. Risulta utile inoltre introdurre la \emph{densità di flusso}---che, purtroppo, viene spesso detto \emph{flusso}  creando non poca confusione---ovvero la grandezza che integrata su tutta una superficie chiusa restituisce il flusso. In questo modo si ha:
    \begin{align}
        L &= \dv{E}{t} = \\mycomma \\
        F &= 
    \end{align}