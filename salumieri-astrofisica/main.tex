\documentclass[leqno, openright, twoside, 11pt]{memoir}
\usepackage{geometry}
\author{P. Salumieri}
\settrims{0pt}{0pt}
\settypeblocksize{*}{1.1\lxvchars}{*}
\setbinding{1.5cm}
\setlrmargins{*}{*}{4}
\setulmarginsandblock{1in}{*}{2}
\checkandfixthelayout

\sideparmargin{outer}
\setmarginnotes{0.1\foremargin}{0.7\foremargin}{\onelineskip}
\aliaspagestyle{chapter}{empty}             % default \pagestyle{chapter} = \pagesstyle{plain} mette il numero al centro in basso.

\setlength{\headwidth}{\textwidth}
    \addtolength{\headwidth}{\marginparsep}
    \addtolength{\headwidth}{\marginparwidth}

\makepagestyle{own}
    \makerunningwidth{own}{\headwidth}
    \makeheadrule{own}{\headwidth}{\normalrulethickness}
    \makeheadposition{own}{flushright}{flushleft}{flushright}{flushleft}
    \makepsmarks{own}{%
        \nouppercaseheads
        \createmark{chapter}{both}{shownumber}{\chaptername\ }{.\ }
        \createmark{section}{right}{shownumber}{}{.\ }
        \createplainmark{toc}{both}{\contentsname}
        \createplainmark{lof}{both}{\listfigurename}
        \createplainmark{lot}{both}{\listtablename}
        \createplainmark{bib}{both}{\bibname}
        \createplainmark{index}{both}{\indexname}
        \createplainmark{glossary}{both}{\glossaryname}    
    }
    \makeevenhead{own}{\bfseries\thepage}{}{\bfseries\leftmark}
    \makeoddhead{own}{\bfseries\rightmark}{}{\bfseries \thepage}
    \makeevenfoot{own}{}{}{\footnotesize \theauthor}
    \makeoddfoot{own}{\footnotesize \theauthor}{}{}
\usepackage[T1]{fontenc}                % per il font
\usepackage[utf8]{inputenc}             % per l'input
\usepackage[english, italian]{babel}    % lingua it

\usepackage{layout}

\usepackage[intlimits]{amsmath}     % =======================================
\usepackage{amssymb}                % intlimits mette gli estremi di integrazione sopra e sotto il segno di integrale
\usepackage{bbm}                    % cose utili per fare matematica e fisica
\usepackage{physics}                % cose utili per fare matematica e fisica
\usepackage[align]{tensor}          % =======================================

\usepackage{graphicx}               % per le foto
\usepackage{tikz}                   % per altre immagini
\usepackage{enumitem}               % per personalizzare gli enumerate
\usepackage{emptypage}              % non se se serve
\usepackage{lettrine}               % per le lettere belle a inizio capitolo
\usepackage{yfonts}                 % per le lettere belle a inizio capitolo
\usepackage{hyperref}               % per fare le cose cliccabili in pdf
    \hypersetup{
        colorlinks=true,
        linkcolor=black,
        filecolor=black,      
        urlcolor=cyan,
        pdfpagemode=FullScreen,
        citecolor=black,
    }
\usepackage{marginnote}
\usepackage[font=footnotesize, hypcap=false]{caption}
\usepackage{lipsum}

\usepackage{xurl}
\usepackage{csquotes}
\usepackage[backend=biber, style=numeric-comp]{biblatex}
    \addbibresource{misc/references.bib}

\usepackage{../../../mypackages/mybookenv}

% ---- REIMPOSTAZIONI ----
\numberwithin{equation}{chapter}                            % Aggiunge il numero del capitolo all'equazione
\setsecnumdepth{subsection}                                 % numera le subsection (1.1.1)

\begin{document}
    \makeatletter
    \pagestyle{own}

    \input{misc/titlepage}

    \frontmatter
        \chapter{Prefazione}
\noindent\yinitpar{\color{purple}T}\textsc{ra i colleghi} circolano diversi blocchi di appunti del corso di Istituzioni di Astrofisica, più o meno ordinati e più o meno completi, prevalentemente divisi in lezioni anziché per argomento. L'intento di questa nuova trascrizione in \LaTeX\ è quello di avere degli appunti più ordinati e organici e meglio impaginati, forza napoli.
    \mainmatter
        \chapter{Introduzione}\label{ch:intro}
    \section{Le dificoltà dell'Astrofisica}
        L'Astrofisica è il ramo della Fisica che studia le proprietà fisiche dei corpi celesti. Essa si deve frequentemente confrontare con distanze di molti oridni  di grandezza speriori a quelle della vita di tutti i giorni e per qesto deve affrontare alcuni problemi che le altre branche della fisica non hanno.

        Mentre un fisico può andare in laboratorio e cambiare le condizioni a contorno del proprio espreimento, l'astrofisico non ha questa possibilità: egli deve dedurre le proprietà dell'oggetto di studio solo tramite osservazioni. Questa differenza richiede che vengano fatte delle ipotesi preliminari senza le quali lo studio dei corpi celesti potrebbe risultare illegittimato.
        
        L'ipotesi su cui si fonda tutta l'Astrofisica è quella che le leggi della fisica siano le stesse ovunque nell'universo---anche ad anni luce di distanza---, cosa che naturalmente non è possibile verificare a meno di, per esempio, di inviare tanti piccoli esploratori in tutti i punti dell'universo per verificare che ciò sia vero.

        Una delle principali difficoltà di questa ipotesi è il fatto che le stesse leggi che applichiamo sulla Terra sono solo modelli che in prima approssimazione descrivono e prevedono sufficientemente bene il mondo che ci circonda e non sono necessariamente corrette. Basti pensare alla Teoria della Gravitazione di Newton che spiega bene l'orbita della Luna intorno alla Terra, ma basta allontanarsi di poco da noi per scoprire che l'orbita di Mercurio ha un comportamento inspiegabile secondo la teoria di Newton ma meglio descritto dalla Relatività Generale.

        L'Astrofisica di conseguenza non può che partire da ipotesi simili---ad esempio assumendo che la Relatività Generale sia vera anche nella galassia di Andromeda---per poi eventualmente confrontarsi con i risultati osservativi e proporre correzioni ai modelli.
        
        In modo del tutto simile, anche le scale temporali sono enormemente più grandi rispetto a quelle della vita dell'uomo. Immaginiamo di guardare diverse foto di una famiglia scattate a distanza di dieci anni l'una dall'altra: in una singola foto saremmo in grado di distinguere le persone anziane da quelle giovani, quelle di sesso maschile da quelle di sesso femminile \myetc, così come confrontando due foto consecutive siamo in grado di riconoscere i cambiamenti nella fisionomia degli individui. Invece immaginiamo di scattare una foto al cielo oggi e un'altra tra dieci anni. Quante differenze saremmo in grado di riconoscere? Quanto cambia una stella nel corso di dieci anni se la sua vita media è dell'ordine di grandezza di miliardi di anni? Si tratta di fenomeni che avvengono su tempi scala troppo lunghi rispetto alla vita umana.

        Il problema dell'Astrofisica è proprio questo: è come tentare di comprendere tutto quello che sappiamo sulla razza umana da una sola foto, di decidere come funzionano le cose semplicemente con uno sguardo attento a un'istantanea, ma non finisce qui! L'informazione infatti si trasferisce con una velocità finita, quindi il segnale proveniente da un oggetto lontano impiegherà più tempo ad arrivare e l'oggetto ci apparirà più giovane. È un po' come se ci venisse chiesto di riconoscere la nonna nella foto di famiglia nonostante appaia come la persona più giovane nella foto.

        Quello che quindi si tenta di fare in Astrofisica è proprio cercare di osservare gli oggetti nel modo più dettagliato possibile per poi cercare di risalire al quadro più generale e estrapolare un modello dell'evoluzione dei corpi celesti.
    \section{Scopi dell'astrofisica}
        %
        %
        %
        %
        %%
        %           SKIP!
        %
        %
        %
        %%
        %
    \section{La misura delle grandezze astronomiche}
        L'Astrofisica trova la sua
        \chapter{Telescopi}
    \epigraph{\itshape C'è una forchetta conficcata nel terreno.}{F. Pinguino}

    \noindent\yinitpar{\color{\initcolor}C}\textsc{ome sappiamo}, la radiazione elettromagnetica è una delle principali fonti di informazioni sugli eventi astronomici e dallo studio della radiazione che giunge sulla Terra dallo spazio è possibile a volte risalire ad alcune proprietà degli eventi astronomici che l'hanno generata.

    Ad esempio, la luce che nel suo tragitto viene riflessa può risultare polarizzata, quella che attraversa gas e polveri può partecipare a fenomeni di scattering e cambiare la propria lunghezza d'onda, così come anche le particelle cariche possono produrre ulteriore radiazione tramite effetto Cherenkov o radiazione di sincrotrone.

    Studiando la radiazione che incide sui nostri strumenti tentiamo quindi di ricostruire il processo fisico che l'ha generata per dedurre le condizioni al contorno che hanno permesso a quel processo di verificarsi.

    Naturalmente questa tecnica ha delle difficoltà legate al fatto che diversi fenomeni possono generare radiazione elettromagnetica simile, e alla limitatezza degli strumenti utilizzati.
    \section{Acquisizione dei dati}
        Per studiare un oggetto nel cielo, la prassi è quella di puntare gli strumenti nella sua direzione e ad un certo istante di tempo misurare l'intensità specifica\footnote{Densità di energia al variare di tutto, vedi Sez xx?}. Per un attimo, immaginiamo di trascurare tutti i fenomeni che alterano la radiazione nel percorso dalla sorgente al rivelatore e concentriamoci sul solo processo di misura.

        Per cominciare, il modo in cui l'osservatore usa lo strumento---che sia consapevole o no delle conseguenze delle proprie scelte---può falsare la misura o causare la perdita di informazioni: se misuro la luminosità di una stella tutti i giorni alla stessa ora e leggo sempre lo stesso valore, potrei essere indotto a pensare che la stella abbia luminosità costante, ma che succede nelle 24 ore di tempo in cui non effettuo misure? La luminosità potrebbe cambiare periodicamente e io potrei aver avuto la ``(s)fortuna'' di aver effettuato le misure in momenti in cui la luminosità assume lo stesso valore, senza pensare che a un orario diverso la luminosità possa essere diversa.

        Assumendo che l'osservatore prenda le misure in modo impeccabile, dovrà comunque scontrarsi con i limiti tecnici dell'apparecchio che ha davanti: il potere risolutivo dell'apparato utilizzato potrebbe non essere sufficiente a risolvere due stelle vicine, inoltre non è detto che esso sia sensibile a tutte le lunghezze d'onda allo stesso modo, così come potrebbe non distinguere lunghezze d'onda vicine. Dovrebbe inoltre essere in grado di misurare la polarizzazione della luce incidente e l'esatto numero di fotoni che incidono sul rivelatore \myetc. Purtroppo uno strumento così versatile ed efficiente non esiste.
    \section{Storia dei telescopi}
    \section{Proprietà geometriche del telescopio}
        \subsection{La distribuzione di Poisson}
    \section{Telescopi rifrattori e riflettori}
    \section{Metodi costruttivi e montature}
    \section{Ottica adattiva}
        \subsection{Seeing}
        \chapter{Fotometria}\label{ch:fotometria}
\epigraph{La fotometria}{ciao}
\noindent\yinitpar{\color{purple}L}\textsc{a fotometria} è lo studio della radiazione elettromagnetica come detto prima bla bla bla. \lipsum[1]
\section{Intensità, flusso, magnitudini}
    Introduciamo le grandezze più importanti della fotometria. Come sappiamo, la radiazione elettromagnetica trasporta un'energia; supponiamo di avere quindi della radiazione che attraversa una superficie $\dd A$ il cui vettore normale forma un angolo $\vartheta$ con la direzione di propagazione. Essa, lasciando la superficie ``alle sue spalle'', si manterrà all'interno di un angolo solido $\dd\Omega$ che stacca dalla normale alla superficie lo stesso angolo $\dd\vartheta$. In generale le radiazione può contenere qualisai lunghezza d'onda: consideriamo inizialmente la radiazione nelle frequenze comprese nell'intervallo  $\bqty{\nu,\nu + \dd\nu}$. L'energia infinitesima che la radiazione trasporta nella regione $\dd\Omega$ sarà quindi $\dd E_\nu \propto \dd t \dd\nu \cos\vartheta\dd A$. Chiamiamo \emph{intensità specifica} la costante di proporzionalità, $I_\nu$, e scriviamo:
    \begin{equation}
        \label{intensità-specifica-freq}
        \dd[4] E_\nu = I_{\nu} \dd{\nu} \dd{t} \cos{\vartheta} \dd{A} \dd{\Omega}
        \myperiod
    \end{equation}
    In modo del tutto analogo possiamo fare lo stesso ragionamento decomponendo lo spettro in lunghezza d'onda anziché in frequenza e avremo:
    \begin{equation}
        \label{intensità-specifica-lambda}
        \dd[4] E_\lambda = I_\lambda \,\dd\lambda\,\dd t\cos\vartheta\,\dd A \,\dd\Omega
        \myperiod
    \end{equation}
    Integrando su tutte le frequenze otteniamo l'\emph{intensità totale} denotata dalla lettera $I$ e data da
    \begin{equation*}
        \label{intensità-totale}
        \dd[3] E = \int_{0}^{\infty}I_\nu \,\dd\nu\,\dd t\cos\vartheta\,\dd A \,\dd\Omega = I \,\dd t\cos\vartheta\,\dd A \,\dd\Omega
        \myperiod
    \end{equation*}
    Invertendo queste relazioni si trova subito:
    \begin{align}
        I_\nu &= \frac{1}{\cos\vartheta}\frac{\dd[4] E_\nu}{\dd{\nu} \dd{t} \dd{A} \dd{\Omega}} \mycomma \\
        I_\lambda &= \frac{1}{\cos\vartheta}\frac{\dd[4] E_\lambda}{\dd{\lambda} \dd{t} \dd{A} \dd{\Omega}}
        \mycomma \\
        I &= \frac{1}{\cos\vartheta}\frac{\dd[3] E}{\dd t \dd{A} \dd{\Omega}}
        = \int_{0}^{\infty}I_\nu \,\dd\nu
        = \int_{0}^{\infty}I_\lambda \,\dd\lambda
        \myperiod
    \end{align}
    Un'altra grandezza utile nell'Astrofisica è il \emph{flusso di energia}, detto atrimenti \emph{flusso} o \emph{luminosità} che coincide con la potenza. Risulta utile inoltre introdurre la \emph{densità di flusso}---che, purtroppo, viene spesso detta \emph{flusso} creando non poca confusione---ovvero la grandezza che integrata su una superficie restituisce il flusso. In questo modo si ha:
    \begin{align}
        \label{eq:lumin-1}
        L &= \dv{E}{t} = \oint\nolimits_{S} F\dd{S} \mycomma \\
        \label{eq:lumin-2}
        L &= 4\pi r^2 F \quad \implies \quad F = \frac{L}{4\pi r^2}
        \mycomma
    \end{align}
    dove l'ultima uguaglianza è ottenuta integrando su una superficie sferica $S$ di raggio $r$ e supponendo $F$ ivi costante.

    Possiamo dedurre che, se la luminosità è una proprietà intrinseca del corpo che emette radiazione---si pensi alla conservazione della potenza nel vuoto, la densità di flusso allora è una grandezza che decresce con $r^2$. Volendo fare un'analogia con l'elettrostatica, $L$ gioca il ruolo della carica\footnote{A rigore, la carica divisa per $\epsilon_{0}$.} netta di una distribuzione contenuta all'interno di una superficie chiusa ed $F$ quello del campo elettrico da essa generata.

    Se un corpo è esteso e non approssimabile come puntiforme, la luminosità e la densità di flusso saranno funzione delle coordinate di ciascun punto del corpo esteso che le genera. Si definisce \emph{brillanza superficiale} la somma (l'integrale) di tutti i contributi di densità di flusso al variare delle sorgenti elementari.
    \subsection{Magnitudine apparente}
        Un primo tentativo di classificazione delle stelle fu fatto da un astronomo di nome Ipparco nel 129 a.C. Egli divise le stelle in sei classi a seconda di quanto apparissero ``brillanti'' a occhio nudo e chiamò queste classi \emph{magnitudini}. Secondo la sua classificazione le stelle più brillanti andavano collocate nella \emph{prima magnitudine}, seguite da quelle di \emph{seconda magnitudine} \myetc, fino a quelle appena visibili che appartenevano alla \emph{sesta magnitudine}.

        Nel 1956, l'astronomo britannico Norman Pogson formalizzò ed estese questa classificazione matematicamente. Pogson si rese conto che il legame tra la densità di flusso di una stella e la sua appartenenza a una cerca classe di magnitudine di Ipparco era tutt'altro che lineare. Supponendo infatti di avere tre stelle i cui \emph{flussi}\footnote{Qui si fa riferimento alla \emph{densità di flusso}. Per brevità anche in questo testo in alcuni punti si userà l'espressione abbreviata. Per non fare confusione, l'usuale \emph{flusso} verrà chiamato sempre \emph{luminosità}.} siano in rapporto $1:10:100$, la differenza di magnitudine tra la prima e la seconda e tra la seconda e la stessa appare la stessa: se la prima stella è di prima magnitudine e la seconda è di terza magnitudine, la terza apparirà di quinta magnitudine. Pogson in particolare notò che a due stelle i cui i flussi sono in un rapporto di $1:100$ corrisponde una differenza di magnitudine pari a $5$; questo vuol dire che a due classi consecutive di magnitudine deve corrispondere un incremento---o decremento---del flusso un fattore $\sqrt[5]{100} \approx \num{2,512}$. Pogson stabilì quindi che la differenza di magnitudine tra due stelle dovesse essere data dalla relazione
        \begin{equation}
            \label{eq:magnitudine-diff}
            m_{2} - m_{1} = -2.5 \Log{\pqty{\frac{F_{2}}{F_{1}}}}
            \mycomma
        \end{equation}
        dove il \num{-2.5} al posto del \num{-2,512} è intenzionale e $\Log \equiv \log_
        {10}$. Naturalmente la \eqref{eq:magnitudine-diff} non permette di definire univocamente la magnitudine apparente di una stella ma solo di valutare la differenza di magnitudine tra due di esse.\footnote{Un po' come il potenziale di una forza che è definito a meno di una costante ma la d.d.p. è univocamente determinata.} Si usa quindi scegliere una certa stella che abbia un certo flusso $F_{0}$ noto a cui viene imposta una magnitudine $m_{0} = 0$; per convenzione questa scelta ricade sulla stella Vega. In questo modo la \eqref{eq:magnitudine-diff} diventa
        \begin{equation}
            \label{magnitudine-app}
            m = -2.5 \Log{\pqty{\frac{F}{F_{0}}}}
            \myperiod
        \end{equation}
        La magnitudine apparente del Sole, avendo posto Vega a $0$, risulta negativa e pari a \num{-26}. Questo significa che il flusso del sole è circa \num{e10} volte quello di Vega. L'oggetto meno luminoso mai misurato possiede invece una magnitudine apparente di \num{30}. Per rendere un'idea di quanta luce provenga effettivamente da Vega, basti pensare che in \num{1}\unit{s} \num{1}\unit{cm^2} di superficie è attraversato da circa \num{900} fotoni in un range di lunghezze d'onda di \num{1}\unit{\angstrom}. Da una lampada invece ne provengono circa \num{e20}.

        È evidente che a magnitudini più basse---o addirittura a valori negativi---corrispondono oggetti più apparentemente luminosi.
    \subsection{Magnitudine assoluta}
        Il fatto che il Sole abbia una magnitudine così ``bassa'' rispetto a tutte le altre stelle non deve indurci a credere che il sole sia effettivamente \num{e10} o più volte più luminoso di esse. Il Sole risulta così fuori scala per via della sua distanza.

        Se teniamo conto del fatto che la magnitudine apparente è definita attraverso il rapporto delle \emph{densità di flusso}, è facile convincersi del fatto che essa non dia alcuna informazione sulla luminosità intrinseca dei corpi celesti considerati. Infatti il flusso decresce col quadrato della distanza e, se immaginiamo di avere stelle identiche a distanze diverse, queste appariranno con magnitudini diverse nonostante la luminosità sia la stessa. Al fine di introdurre un modo più sistematico di valutare la luminosità delle stelle, immaginiamo di prendere tutte le stelle dell'Universo e metterle alla stessa distanza dal nostro punto di osservazione. A questo punto la differenza di magnitudine tra due stelle sarà coerente con la differenza delle loro luminosità---e anche del loro flusso, vista la \eqref{eq:lumin-2}: una stella più luminosa avrà un flusso maggiore e una magnitudine ``più negativa'' e, viceversa, a stelle meno luminose corrisponderanno magnitudini più alte.
        
        Se decidiamo di scegliere questa distanza pari a \num{10}\unit{\parsec}, otteniamo quella che viene detta \emph{magnitudine assoulta}. La magnitudine assoluta si indica con la lettera $M$ e lega la magnitudine apparente di una stella alla sua distanza da noi. Infatti se indichiamo con $F_r$ il flusso di una stella a distanza $r$ e con $F_{10}$ il suo flusso a distanza \num{10}\unit{pc}, si ha dalla \eqref{eq:lumin-2}
        \begin{equation*}
            \frac{F_r}{F_{10}} = \frac{4 \pi \pqty{\num{10}\unit{\parsec}}^2}{4 \pi r^2} = \pqty{\frac{\num{10}\unit{\parsec}}{r}}^2
        \end{equation*}
        che, inserito nella \eqref{eq:magnitudine-diff} dà
        \begin{equation*}
            m - M = -2.5 \Log{\pqty{\frac{\num{10}\unit{\parsec}}{r}}}^2
            \myperiod
        \end{equation*}
        Portando un $-2$ fuori dal logaritmo si ottiene quello che viene detto \emph{modulo della distanza}:
        \begin{equation}
            \label{eq:modulo-distanza}
            m - M = 5 \Log{\pqty{\frac{r}{\num{10}\unit{\parsec}}}}
        \end{equation}
    \subsection{Magnitudine fotografica e bolometrica}
        Fin'ora nel parlare di magnitudini abbiamo completamente trascurato un'informazione importante. Il flusso di una stella dovrebbe essere calcolato tenendo conto di tutti i fotoni ricevuti a tutte le lunghezze d'onda, cosa che di certo l'occhio umano non può fare.
        
        Potremmo provare a migliorare la nostra classificazione con l'aiuto di un sensore \ccd, che ha un'efficienza quantica superiore a quella dell'occhio umano e permette di fare anche esposizioni più lunghe migliorando il rapporto segnale--rumore. Il primo a fare un tentativo simile fu l'astronomo G. P. Bond nel XIX Secolo che ebbe l'intuizione di porre una lastra fotografica sul piano focale di un telescopio. Fotografando diversi oggetti notò che quelli più luminosi lasciavano sulla lastra delle ``macchie'' più grandi e questo permetteva di asssegnare loro una magnitudine in modo più sistematico. Una magnitudine definita in questo modo può risultare sufficientemente utile da avere un nome suo, quello di \emph{magnitudine fotografica}.

        Se provassimo a fare la stessa cosa, con una \ccd---ma anche con una lastra fotografica---dopo poche misurazioni inizieremmo a notare delle discrepanze tra le magnitudini fotografiche e quelle \emph{visuali}:\footnote{Dall'osservazione a occhio nudo.} due stelle con magnitudine visuale uguale potrebbero avere magnitudine fotografica differente e viceversa, oppure nella foto potrebbero comparire stelle invisibili a occhio nudo. Questo è dovuto al fatto che l'efficienza quantica dei sensori \ccd\ non solo è in generale più elevata, ma lo è in un intervallo più esteso di quello dell'occhio umano, con un picco leggermente spostato verso il blu rispetto a quello dell'occhio che ricade nel verde.

        Si definisce invece \emph{magnitudine bolometrica} la magnitudine teorica ottenuta integrando segnale su tutte le lunghezze d'onda alla massima efficienza. Si tratta di una idealizzazione impossibile da ottenere con le misure a causa dei limiti pratici dovuti al mezzo interstellare, l'atmosfera, l'inefficienza degli strumenti \myetc.
\section{Il sistema \ubvri}
        \chapter{Spettroscopia}
    La fotometria ci permette di misurare la radiazione elettromagnetica in bande larghe circa \SI{1500}{\angstrom}. Questo ci permette di risalire alle proprietà generali della sorgente che la emette, come la temperatura e la distanza. Tuttavia, come abbiamo visto, tra le sorgenti e i nostri strumenti si trovano oggetti materiali che la luce deve attraversare per raggiungerci.
    
    Proviamo per un attimo a fare un ragionamento al contrario: immaginiamo di avere una sorgente di cui conosciamo tutte le caratteristiche principali---ad esempio proprio un corpo nero---e interponiamo del materiale tra noi e la sorgente. Possiamo ad esempio confrontare la luce che arriva ai nostri strumenti con la curva teorica prodotta dal corpo nero al fine capire come la radiazione interagisce con la materia e dedurre da ciò le sue proprietà.
\section{Assorbimento ed emissione}
    La materia che la radiazione elettromagnetica attraversa dopo aver lasciato la stella è di vario tipo: polveri e grani, gas ionizzati, plasmi, e le stesse atmosfere della stella e del nostro pianeta, ammesso che i nostri strumenti non si trovino nello spazio.

    I fenomeni di interazione che possono verificarsi sono prevalentemente di tre tipi:
    \begin{enumerate}[label=\ding{70}]
        \item \emph{vero assorbimento}, quando il fotone di energia $h\nu$ viene del tutto ``assorbito'' da un atomo che si eccita e la sua energia si trasforma in energia termodinamica;
        \item \emph{vera emissione}, quando un atomo eccitato---da assorbimento o a causa di urti anelastici con altri atomi---emette radiazione diseccitandosi;
        \item \emph{diffusione Compton}, un fenomeno anelastico in cui un fotone interagisce con un elettrone libero, gli cede energia cinetica e viene diffuso a un angolo $\vartheta$ cambiando lunghezza d'onda.
    \end{enumerate}
    Questi sono fenomeni che possono alterare il 
\section{Spettrografi}
    Per poter misurare la radiazione incidente sugli strumenti con le
        \input{chapters/stelle}
        \input{chapters/sole}

    \backmatter
        \printbibliography

\end{document}