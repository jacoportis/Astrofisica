\documentclass[leqno, openright, twoside, 11pt]{memoir}
\usepackage{geometry}
\author{P. Salumieri}
\settrims{0pt}{0pt}
\settypeblocksize{*}{1.1\lxvchars}{*}
\setbinding{1.5cm}
\setlrmargins{*}{*}{4}
\setulmarginsandblock{1in}{*}{2}
\checkandfixthelayout

\sideparmargin{outer}
\setmarginnotes{0.1\foremargin}{0.7\foremargin}{\onelineskip}
\aliaspagestyle{chapter}{empty}             % default \pagestyle{chapter} = \pagesstyle{plain} mette il numero al centro in basso.

\setlength{\headwidth}{\textwidth}
    \addtolength{\headwidth}{\marginparsep}
    \addtolength{\headwidth}{\marginparwidth}

\makepagestyle{own}
    \makerunningwidth{own}{\headwidth}
    \makeheadrule{own}{\headwidth}{\normalrulethickness}
    \makeheadposition{own}{flushright}{flushleft}{flushright}{flushleft}
    \makepsmarks{own}{%
        \nouppercaseheads
        \createmark{chapter}{both}{shownumber}{\chaptername\ }{.\ }
        \createmark{section}{right}{shownumber}{}{.\ }
        \createplainmark{toc}{both}{\contentsname}
        \createplainmark{lof}{both}{\listfigurename}
        \createplainmark{lot}{both}{\listtablename}
        \createplainmark{bib}{both}{\bibname}
        \createplainmark{index}{both}{\indexname}
        \createplainmark{glossary}{both}{\glossaryname}    
    }
    \makeevenhead{own}{\bfseries\thepage}{}{\bfseries\leftmark}
    \makeoddhead{own}{\bfseries\rightmark}{}{\bfseries \thepage}
    \makeevenfoot{own}{}{}{\footnotesize \theauthor}
    \makeoddfoot{own}{\footnotesize \theauthor}{}{}
\usepackage[T1]{fontenc}                % per il font
\usepackage[utf8]{inputenc}             % per l'input
\usepackage[english, italian]{babel}    % lingua it

\usepackage{layout}
\usepackage{lipsum}

\usepackage[intlimits]{amsmath}     % =======================================
\usepackage{amssymb}                % intlimits mette gli estremi di integrazione sopra e sotto il segno di integrale
\usepackage{bbm}                    % cose utili per fare matematica e fisica
\usepackage{siunitx}
\usepackage{physics}                % cose utili per fare matematica e fisica
\usepackage[align]{tensor}          % =======================================


\usepackage{graphicx}               % per le foto
\usepackage{tikz}                   % per altre immagini
\usepackage{enumitem}               % per personalizzare gli enumerate
\usepackage{emptypage}              % non se se serve
\usepackage{lettrine}               % per le lettere belle a inizio capitolo
\usepackage{yfonts}                 % per le lettere belle a inizio capitolo
\usepackage{hyperref}               % per fare le cose cliccabili in pdf
    \hypersetup{
        colorlinks=true,
        linkcolor=black,
        filecolor=black,      
        urlcolor=cyan,
        pdfpagemode=FullScreen,
        citecolor=black,
    }
\usepackage{marginnote}
\usepackage[font=footnotesize, hypcap=false]{caption}
\usepackage{lipsum}

\usepackage{xurl}
\usepackage{csquotes}
\usepackage[backend=biber, style=numeric-comp]{biblatex}
    \addbibresource{misc/references.bib}

\usepackage{misc/mybookenv}

% ---- REIMPOSTAZIONI ----
\numberwithin{equation}{chapter}                            % Aggiunge il numero del capitolo all'equazione
\setsecnumdepth{subsection}                                 % numera le subsection (1.1.1)

% ---- COMANDI DI QUESTO LIBRO ----
\DeclareSIUnit\angstrom{\text{\AA}}
\DeclareSIUnit\parsec{\text{pc}}
\DeclareSIUnit\percent{\%}

\newcommand{\ccd}{CCD}
\newcommand{\ubvri}{UBVRI}

\begin{document}
    \makeatletter
    \pagestyle{own}

    \input{misc/titlepage}

    \frontmatter
        \tableofcontents
        \chapter{Prefazione}
\noindent\yinitpar{\color{purple}T}\textsc{ra i colleghi} circolano diversi blocchi di appunti del corso di Istituzioni di Astrofisica, più o meno ordinati e più o meno completi, prevalentemente divisi in lezioni anziché per argomento. L'intento di questa nuova trascrizione in \LaTeX\ è quello di avere degli appunti più ordinati e organici e meglio impaginati, forza napoli.
    \mainmatter
        \chapter{Introduzione}\label{ch:intro}
    \section{Le dificoltà dell'Astrofisica}
        L'Astrofisica è il ramo della Fisica che studia le proprietà fisiche dei corpi celesti. Essa si deve frequentemente confrontare con distanze di molti oridni  di grandezza speriori a quelle della vita di tutti i giorni e per qesto deve affrontare alcuni problemi che le altre branche della fisica non hanno.

        Mentre un fisico può andare in laboratorio e cambiare le condizioni a contorno del proprio espreimento, l'astrofisico non ha questa possibilità: egli deve dedurre le proprietà dell'oggetto di studio solo tramite osservazioni. Questa differenza richiede che vengano fatte delle ipotesi preliminari senza le quali lo studio dei corpi celesti potrebbe risultare illegittimato.
        
        L'ipotesi su cui si fonda tutta l'Astrofisica è quella che le leggi della fisica siano le stesse ovunque nell'universo---anche ad anni luce di distanza---, cosa che naturalmente non è possibile verificare a meno di, per esempio, di inviare tanti piccoli esploratori in tutti i punti dell'universo per verificare che ciò sia vero.

        Una delle principali difficoltà di questa ipotesi è il fatto che le stesse leggi che applichiamo sulla Terra sono solo modelli che in prima approssimazione descrivono e prevedono sufficientemente bene il mondo che ci circonda e non sono necessariamente corrette. Basti pensare alla Teoria della Gravitazione di Newton che spiega bene l'orbita della Luna intorno alla Terra, ma basta allontanarsi di poco da noi per scoprire che l'orbita di Mercurio ha un comportamento inspiegabile secondo la teoria di Newton ma meglio descritto dalla Relatività Generale.

        L'Astrofisica di conseguenza non può che partire da ipotesi simili---ad esempio assumendo che la Relatività Generale sia vera anche nella galassia di Andromeda---per poi eventualmente confrontarsi con i risultati osservativi e proporre correzioni ai modelli.
        
        In modo del tutto simile, anche le scale temporali sono enormemente più grandi rispetto a quelle della vita dell'uomo. Immaginiamo di guardare diverse foto di una famiglia scattate a distanza di dieci anni l'una dall'altra: in una singola foto saremmo in grado di distinguere le persone anziane da quelle giovani, quelle di sesso maschile da quelle di sesso femminile \myetc, così come confrontando due foto consecutive siamo in grado di riconoscere i cambiamenti nella fisionomia degli individui. Invece immaginiamo di scattare una foto al cielo oggi e un'altra tra dieci anni. Quante differenze saremmo in grado di riconoscere? Quanto cambia una stella nel corso di dieci anni se la sua vita media è dell'ordine di grandezza di miliardi di anni? Si tratta di fenomeni che avvengono su tempi scala troppo lunghi rispetto alla vita umana.

        Il problema dell'Astrofisica è proprio questo: è come tentare di comprendere tutto quello che sappiamo sulla razza umana da una sola foto, di decidere come funzionano le cose semplicemente con uno sguardo attento a un'istantanea, ma non finisce qui! L'informazione infatti si trasferisce con una velocità finita, quindi il segnale proveniente da un oggetto lontano impiegherà più tempo ad arrivare e l'oggetto ci apparirà più giovane. È un po' come se ci venisse chiesto di riconoscere la nonna nella foto di famiglia nonostante appaia come la persona più giovane nella foto.

        Quello che quindi si tenta di fare in Astrofisica è proprio cercare di osservare gli oggetti nel modo più dettagliato possibile per poi cercare di risalire al quadro più generale e estrapolare un modello dell'evoluzione dei corpi celesti.
    \section{Scopi dell'astrofisica}
        %
        %
        %
        %
        %%
        %           SKIP!
        %
        %
        %
        %%
        %
    \section{La misura delle grandezze astronomiche}
        L'Astrofisica trova la sua
        \chapter{Telescopi}
    \epigraph{\itshape C'è una forchetta conficcata nel terreno.}{F. Pinguino}
    \section{Storia dei telescopi}
    \section{Proprietà geometriche del telescopio}
        \subsection{La distribuzione di Poisson}
    \section{Telescopi rifrattori e riflettori}
    \section{Metodi costruttivi e montature}
    \section{Ottica adattiva}
        \subsection{Seeing}
        \chapter{Fotometria}\label{ch:fotometria}
\epigraph{La fotometria}{ciao}
\noindent\yinitpar{\color{purple}L}\textsc{a fotometria} è lo studio della radiazione elettromagnetica come detto prima bla bla bla. \lipsum[1]
\section{Intensità, flusso, magnitudini}.
    Introduciamo le grandezze più importanti della fotometria. Come sappiamo, la radiazione elettromagnetica trasporta un'energia; supponiamo di avere quindi della radiazione che attraversa una superficie $\dd A$ il cui vettore normale forma un angolo $\vartheta$ con la direzione di propagazione. Essa, lasciando la superficie ``alle sue spalle'', si manterrà all'interno di un angolo solido $\dd\Omega$ che stacca dalla normale alla superficie lo stesso angolo $\dd\vartheta$. In generale le radiazione può contenere qualisai lunghezza d'onda: consideriamo inizialmente la radiazione nelle frequenze comprese nell'intervallo  $\bqty{\nu,\nu + \dd\nu}$. L'energia infinitesima che la radiazione trasporta nella regione $\dd\Omega$ sarà quindi $\dd E_\nu \propto \dd t \dd\nu \cos\vartheta\dd A$. Chiamiamo \emph{intensità specifica} la costante di proporzionalità, $I_\nu$, e scriviamo:
    \begin{equation}
        \label{intensità-specifica-freq}
        \dd[4] E_\nu = I_{\nu} \dd{\nu} \dd{t} \cos{\vartheta} \dd{A} \dd{\Omega}
        \myperiod
    \end{equation}
    In modo del tutto analogo possiamo fare lo stesso ragionamento decomponendo lo spettro in lunghezza d'onda anziché in frequenza e avremo:
    \begin{equation}
        \label{intensità-specifica-lambda}
        \dd[4] E_\lambda = I_\lambda \,\dd\lambda\,\dd t\cos\vartheta\,\dd A \,\dd\Omega
        \myperiod
    \end{equation}
    Integrando su tutte le frequenze otteniamo l'\emph{intensità totale} denotata dalla lettera $I$ e data da
    \begin{equation*}
        \label{intensità-totale}
        \dd[3] E = \int_{0}^{\infty}I_\nu \,\dd\nu\,\dd t\cos\vartheta\,\dd A \,\dd\Omega = I \,\dd t\cos\vartheta\,\dd A \,\dd\Omega
        \myperiod
    \end{equation*}
    Invertendo queste relazioni si trova subito:
    \begin{align}
        I_\nu &= \frac{1}{\cos\vartheta}\frac{\dd[4] E_\nu}{\dd{\nu} \dd{t} \dd{A} \dd{\Omega}} \mycomma \\
        I_\lambda &= \frac{1}{\cos\vartheta}\frac{\dd[4] E_\lambda}{\dd{\lambda} \dd{t} \dd{A} \dd{\Omega}}
        \mycomma \\
        I &= \frac{1}{\cos\vartheta}\frac{\dd[3] E}{\dd t \dd{A} \dd{\Omega}}
        \myperiod
    \end{align}
    Un'altra grandezza utile nell'Astrofisica è il \emph{flusso di energia}, detto atrimenti \emph{flusso} o \emph{luminosità} che coincide con la potenza. Risulta utile inoltre introdurre la \emph{densità di flusso}---che, purtroppo, viene spesso detta \emph{flusso} creando non poca confusione---ovvero la grandezza che integrata su una superficie restituisce il flusso. In questo modo si ha:
    \begin{align}
        L &= \dv{E}{t} = \oint\nolimits_{S} F\dd{S} \mycomma \\
        L &= 4\pi r^2 F \quad \implies \quad F = \frac{L}{4\pi r^2}
        \mycomma
    \end{align}
    dove l'ultima uguaglianza è ottenuta integrando su una superficie sferica $S$ di raggio $r$ e supponendo $F$ ivi costante.

    Possiamo dedurre che, se la luminosità è una proprietà intrinseca del corpo che emette radiazione---si pensi alla conservazione della potenza nel vuoto, la densità di flusso allora è una grandezza che decresce con $r^2$. Volendo fare un'analogia con l'elettrostatica, $L$ gioca il ruolo della carica\footnote{A rigore, la carica divisa per $\epsilon_{0}$.} netta di una distribuzione contenuta all'interno di una superficie chiusa ed $F$ quello del campo elettrico da essa generata.

    Se un corpo è esteso e non approssimabile come puntiforme, la luminosità e la densità di flusso saranno funzione delle coordinate di ciascun punto del corpo esteso che le genera. Si definisce \emph{brillanza superficiale} la somma (l'integrale) di tutti i contributi di densità di flusso al variare delle sorgenti elementari.
    \subsection{Magnitudine apparente}
        Un primo tentativo di classificazione delle stelle fu fatto da un astronomo di nome Ipparco nel 129 a.C. Egli divise le stelle in sei classi a seconda di quanto apparissero ``brillanti'' a occhio nudo e chiamò queste classi \emph{magnitudini}. Secondo la sua classificazione le stelle più brillanti andavano collocate nella \emph{prima magnitudine}, seguite da quelle di \emph{seconda magnitudine} \myetc, fino a quelle appena visibili che appartenevano alla \emph{sesta magnitudine}.

        Nel 1956, l'astronomo britannico Norman Pogson formalizzò ed estese questa classificazione matematicamente. Pogson si rese conto che il legame tra la densità di flusso di una stella e la sua appartenenza a una cerca classe di magnitudine di Ipparco era tutt'altro che lineare. Supponendo infatti di avere tre stelle i cui \emph{flussi}\footnote{Qui si fa riferimento alla \emph{densità di flusso}. Per brevità anche in questo testo da adesso si userà l'espressione abbreviata. Per non fare confusione, l'usuale \emph{flusso} verrà chiamato sempre \emph{luminosità}.} siano in rapporto $1:10:100$, la differenza di magnitudine tra la prima e la seconda e tra la seconda e la stessa appare la stessa: se la prima stella è di prima magnitudine e la seconda è di terza magnitudine, la terza apparirà di quinta magnitudine. Pogson in particolare notò che a due stelle i cui i flussi sono in un rapporto di $1:100$ corrisponde una differenza di magnitudine pari a $5$; questo vuol dire che a due classi consecutive di magnitudine deve corrispondere un incremento---o decremento---del flusso un fattore $\sqrt[5]{100} \approx \num{2,512}$. Pogson stabilì quindi che la differenza di magnitudine tra due stelle dovesse essere data dalla relazione
        \begin{equation}
            \label{eq:magnitudine-diff}
            m_{2} - m_{1} = -2.5 \Log{\pqty{\frac{F_{2}}{F_{1}}}}
            \mycomma
        \end{equation}
        dove il $-2.5$ al posto del $\num{-2,512}$ è intenzionale e $\Log \equiv \log_
        {10}$. Naturalmente la \eqref{eq:magnitudine-diff} non permette di definire univocamente la magnitudine apparente di una stella ma solo di valutare la differenza di magnitudine tra due di esse.\footnote{Un po' come il potenziale di una forza che è definito a meno di una costante ma la d.d.p. è univocamente determinata.} Si usa quindi scegliere una certa stella che abbia un certo flusso $F_{0}$ noto a cui viene imposto $m_{0} = 0$; per convenzione questa scelta ricade sulla stella Vega. In questo modo la \eqref{eq:magnitudine-diff} diventa
        \begin{equation*}
            m_{2} = -2.5 \Log{\pqty{\frac{F}{F_{0}}}}
            \myperiod
        \end{equation*}
        \chapter{Spettroscopia}
\section{Radiazione di corpo nero}
    Il corpo nero è un'utile astrazione fisica ed è definito come un corpo che assorbe tutta la radiazione incidente senza rifletterla. Una buona rappresentazione di un corpo nero è data da un guscio sferico a una temperatura $T$ fissata e completamente isolato. Il guscio emetterà radiazione 
\section{La tomo di drogeno}
        \input{chapters/stelle}
        \input{chapters/sole}

    \backmatter
        \printbibliography

\end{document}