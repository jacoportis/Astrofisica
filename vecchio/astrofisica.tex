\documentclass[openany,12pt]{article}
\usepackage[utf8]{inputenc}
\usepackage[letterpaper,top=2cm,bottom=2cm,left=3cm,right=3cm,marginparwidth=1.75cm]{geometry}
\usepackage{wrapfig}
\usepackage{hyperref}
    \hypersetup{
        colorlinks=true,
        linkcolor=black,
        filecolor=black,      
        urlcolor=cyan,
        pdfpagemode=FullScreen,
        citecolor=black,
    }
\usepackage[italian]{babel}
\usepackage{epigraph}
\usepackage{afterpage}
\newcommand\blankpage{%
    \null
    \thispagestyle{empty}%
    \newpage}
\usepackage{import}
\usepackage{amsfonts}
\usepackage{graphicx}
\usepackage{amssymb}
\usepackage{amsmath}
\usepackage{physics}
\usepackage[version=4]{mhchem}
\usepackage[makeroom]{cancel}
\newcommand{\myol}[2][3]{{}\mkern#1mu\overline{\mkern-#1mu#2}}
\newcommand{\notimplies}{%
  \mathrel{{\ooalign{\hidewidth$\not\phantom{=}$\hidewidth\cr$\implies$}}}}
  \newcommand{\notimpliedby}{%
  \mathrel{{\ooalign{\hidewidth$\not\phantom{=}$\hidewidth\cr$\impliedby$}}}}
%\usepackage{unicode-math}
\DeclareRobustCommand{\rifin}{\text{\reflectbox{$\in$}}}

\usepackage{enumitem}
\usepackage{array}
\usepackage{tikz}
\usetikzlibrary{arrows.meta}
\usepackage{pgfplots}
\usetikzlibrary{shapes}
\usepackage{float}
\newcommand*\circled[1]{\tikz[baseline=(char.base)]{
            \node[shape=circle,draw,inner sep=2pt] (char) {#1};}}
\setlength\parindent{0pt}%e si gode, toglie lo spostmento a destra di una nuova riga
\usepackage{caption}
\usepackage{subcaption}
\newcommand{\comment}[1]{}

%simboli
\usepackage{halloweenmath}
\usepackage{pifont}
\newcommand{\E}{È \hspace{0.1mm}}

\newcommand{\A}{\text{Å \hspace{0.1mm}}}

\begin{document}

  \thispagestyle{empty}
  \begin{center}

  \begin{minipage}[c]{0.45\textwidth}
  \begin{flushleft}
  \includegraphics[width=0.8\textwidth]{logo-unict-orizzontale-grigio.png}
  \end{flushleft}
  \end{minipage}
  \hfill
  \begin{minipage}[c]{0.45\textwidth}
  \begin{flushright}
  \includegraphics[width=\textwidth]{logo_dfa_orizzontale}
  \end{flushright}
  \end{minipage}\\
  \medskip
  \hbox to \textwidth{\hrulefill}

  \vfill
  \vfill


  \uppercase{\sc{ \Large{\textbf{Astrofisica}}}}\\

  \vfill
  \large{A cura di Peppino Salumieri}

  \vfill
  \vfill
  \hbox to \textwidth{\hrulefill}
  {\sc anno 2024}
  \end{center}

  \newpage

  \blankpage

  \tableofcontents

  \newpage

  \blankpage

  \section*{Guida alla lettura}

  I presenti appunti sono basati sulle lezioni del prof. Leone, ma sono stati tuttavia parecchio rivisitati integrando da vari libri. Si sconsiglia quindi di usarla come unica fonte.

  In alcune parti si trovano dei paragrafi scritti tra due linee, così:

  \hrulefill

  \begin{center}
    \textsc{Esempio di calcoli astro-meccanici}
  \end{center}

  \hrulefill

  i quali sono degli approfondimenti, aggiunti da me per chiarire meglio i concetti.

  Un immenso ringraziamento va a Marika Buccheri, che ha fornito parecchi chiarimenti.

  \newpage

  \section{Introduzione}

  \subsection{Fisica vs astrofisica}
    \import{./Chapter/1-Sections/}{1-primo}

    \subsection{Scopi dell'astrofisica}
    \import{./Chapter/1-Sections/}{2-secondo}

    \subsection{Tempo astronomico}
    \import{./Chapter/1-Sections/}{3-terzo}

    \subsection{Coordinate astronomiche}
    \import{./Chapter/1-Sections/}{4-quarto}

  \newpage

  \section{La radiazione elettromagnetica}

    \subsection{Grandezze utilizzate}
    \import{./Chapter/2-Sections/}{1-primo}

    \subsection{Tecniche astronomiche: fotometria}
    \import{./Chapter/2-Sections/}{2-secondo}

  \newpage

  \section{Telescopi}

    \subsection{Come funziona un telescopio}
    \import{./Chapter/3-Sections/}{1-primo}

    \subsection{I problemi dei telescopi}
    \import{./Chapter/3-Sections/}{2-secondo}

    \subsection{Le varie tipologie di telescopi}
    \import{./Chapter/3-Sections/}{3-terzo}

  \newpage

  \section{Fotometria}

    \subsection{Perché è importante riconoscere i tipi di radiazione}
    \import{./Chapter/4-Sections/}{1-primo}

  \newpage

  \section{Spettroscopia}
    %subsection dentro il file
    \import{./Chapter/5-Sections/}{1-primo}

    \subsection{Trasporto radiativo}
    \import{./Chapter/5-Sections/}{2-secondo}

    \subsection{L'allargamento delle righe spettrali. Profilo del coefficiente di assorbimento}
    \import{./Chapter/5-Sections/}{3-terzo}

    \subsection{L'origine e l'intensità delle righe spettrali}
    \import{./Chapter/5-Sections/}{4-quarto}

    \subsection{L'opacità al continuo}
    \import{./Chapter/5-Sections/}{5-quinto}

  \newpage

  \section{Atmosfere stellari}
    \import{./Chapter/6-Sections/}{1-primo}

  \newpage

  \section{Il Sole}
    \subsection{Ora viremu}
    \import{./Chapter/7-Sections/}{1-primo}
\end{document}
