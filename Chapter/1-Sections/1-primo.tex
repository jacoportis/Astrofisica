L'astrofisica deve affrontare alcuni problemi che le altre branche della fisica non hanno.

Mentre un fisico può andare in laboratorio e cambiare le condizioni a contorno del proprio campione, l'astrofisico non ha questa possibilità: egli deve dedurre le proprietà del proprio oggetto solo osservandolo. Questa differenza prevede delle ipotesi, cioè si suppone che le leggi della fisica che valgono sulla Terra valgano ovunque.

Ma tale ipotesi è vera? Per esempio abbiamo applicato la gravità di Newton per secoli, ma quando l'abbiamo applicata all'orbita di Mercurio si è visto che non era valida perché ha deformazioni relativistiche.

Da tale esempio deduciamo che quando facciamo delle ipotesi per studiare un oggetto dell'esterno che stiamo osservando dobbiamo chiederci se tali ipotesi siano verosimili, e per ovviare a questo problema abbiamo detto che le leggi della fisica valgono ovunque. Tale supposizione permette di verificare la validità delle teorie.

\vspace{0.2cm}Un altro problema è che si osservano oggetti che non cambiano nella generalità: se facciamo diverse foto del cielo esse risultano essere uguali, anche a distanza di parecchi anni.

Immaginiamo di guardare la fotografia di una famiglia: saremmo in grado di distinguere le persone anziane da quelle giovani, quelle di sesso maschile da quelle di sesso femminile ecc. Siamo in grado di farlo perché sappiamo tutto sulla fisiologia umana: abbiamo chiaro ad esempio il processo di invecchiamento, di evoluzione. Se ora guardassimo una foto dello spazio, cosa sapremmo dire?

Il problema dell'astrofisica è proprio questo: è come se si dovesse comprendere tutto quello che sappiamo sulla razza umana da una sola foto, cioè si tratta di decidere come funzionano le cose semplicemente guardando. In altre parole, dovremmo interpretare fenomeni che avvengono su tempi scala troppo lunghi rispetto alla vita umana.

In realtà il problema è ancora più complicato: l'informazione si trasferisce con una velocità finita, quindi l'informazione proveniente da un oggetto lontano necessità di più tempo e di conseguenza a noi l'oggetto appare più giovane. Se volessimo fare un analogo, è come se ci mostrassero una foto della nonna da bambina e da questa dovessimo dedurre che è la nonna.

\vspace{0.2cm}La vera difficoltà dell'astrofisica è, dunque, mettere insieme tutte le informazioni e fare un quadro in evoluzione con oggetti diversi. Questo si fa studiando alcuni oggetti molto in dettaglio.